\documentclass{article}
\begin{document}
This is my \emph{first} document prepared in \LaTeX. I typed it on \today.
\indent We have seen that to typeset something in \LaTeX, we type in the
text to be typeset together with some \LaTeX\ commands.
Words must be separated by spaces (does not matter how many)
and lines maybe broken arbitrarily.\\
\indent The end of a paragraph is specified by a \emph{blank line}
in the input. In other words, whenever you want to start a new
paragraph, just leave a blank line and proceed.\\
\indent The numbers 1, 2, 3, etc. are called natural numbers. According to
Kronecker, they were made by God;\ all else being the works of Man.

"Note" the difference in right and left quotes in \lq single
quotes\rq\ and \lq\lq double quotes\rq\rq.\\
\indent X-rays are discussed in pages 221--225 of Volume 3---the volume on
electromagnetic waves.

Maybe I have now learnt about 1\% of \LaTeX.\\
This is the first line.\\[10pt] 
This is the second line\\
--------------------------------------------------\\
\begin{center}
	The \TeX nical Institute\\[.75cm]
	Certificate
\end{center}
\noindent This is to certify that Mr. N. O. Vice has undergone a
course at this institute and is qualified to be a \TeX nician.
\begin{flushright}
	The Director\\
	The \TeX nical Institute
\end{flushright} 
\begin{center}
	FONTS
\end{center}
\noindent \textit{A polygon of three sides is called a \emph{triangle} and a
polygon of four sides is called a \emph{quadrilateral}}\\
\textbf{A polygon of three sides is called a
	\emph{triangle} and a polygon of four sides is called a
	\emph{quadrilateral}}\\
    A is \textbf{A}, B is B\\
------------------------------ \textbf{Split\ Line}---------------------------------
\begin{center}
	{\bfseries\huge The \TeX nical Institute}\\[1cm]
	{\scshape\LARGE Certificate}
\end{center}
\noindent This is to certify that Mr. N. O. Vice has undergone a
course at this institute and is qualified to be a \TeX nical Expert.
\begin{flushright}
	{\sffamily The Director\\
		The \TeX nical Institute}
\end{flushright}


































\end{document}