\documentclass{article}
\usepackage{pifont}
\usepackage{amssymb}
\usepackage{array}
\usepackage{multirow}
\usepackage{amsmath}
\usepackage{mathrsfs}
\usepackage{amsthm}
\usepackage{graphicx}
\usepackage{subfigure}
\usepackage[includemp,body={398pt,550pt},footskip=30pt,%
marginparwidth=60pt,marginparsep=10pt]{geometry}
\graphicspath{{C:\Users\hou\Desktop\Baidu Map\1\example\Latex_Tutor\picture}}
\newtheorem{thm}{Theorem}[section]
\theoremstyle{definition}
\newtheorem{dfn}{Definition}
\theoremstyle{remark}
\newtheorem{note}{Note}[section]
\theoremstyle{plain}
\newtheorem{lem}[thm]{Lemma}
\newcommand{\halmos}{\rule{1mm}{2.5mm}}
\renewcommand{\qedsymbol}{\halmos}
\numberwithin{equation}{subsection}
\graphicspath{C:\Users\hou\Desktop\Baidu Map\1\example\Latex_Tutor\picture}
\begin{document}
This is my \emph{first} document prepared in \LaTeX. I typed it on \today.
\indent We have seen that to typeset something in \LaTeX, we type in the
text to be typeset together with some \LaTeX\ commands.
Words must be separated by spaces (does not matter how many)
and lines maybe broken arbitrarily.\\
\indent The end of a paragraph is specified by a \emph{blank line}
in the input. In other words, whenever you want to start a new
paragraph, just leave a blank line and proceed.\\
\indent The numbers 1, 2, 3, etc. are called natural numbers. According to
Kronecker, they were made by God;\ all else being the works of Man.

"Note" the difference in right and left quotes in \lq single
quotes\rq\ and \lq\lq double quotes\rq\rq.\\
\indent X-rays are discussed in pages 221--225 of Volume 3---the volume on
electromagnetic waves.

Maybe I have now learnt about 1\% of \LaTeX.\\
This is the first line.\\[10pt] 
This is the second line\\
--------------------------------------------------\\
\begin{center}
	The \TeX nical Institute\\[.75cm]
	Certificate
\end{center}
\noindent This is to certify that Mr. N. O. Vice has undergone a
course at this institute and is qualified to be a \TeX nician.
\begin{flushright}
	The Director\\
	The \TeX nical Institute
\end{flushright} 
\begin{center}
	FONTS
\end{center}
\noindent \textit{A polygon of three sides is called a \emph{triangle} and a
polygon of four sides is called a \emph{quadrilateral}}\\
\textbf{A polygon of three sides is called a
	\emph{triangle} and a polygon of four sides is called a
	\emph{quadrilateral}}\\
    A is \textbf{A}, B is B\\
------------------------------ \textbf{Split\ Line}---------------------------------
\begin{center}
	{\bfseries\huge The \TeX nical Institute}\\[1cm]
	{\scshape\LARGE Certificate}
\end{center}
\noindent This is to certify that Mr. N. O. Vice has undergone a
course at this institute and is qualified to be a \TeX nical Expert.
\begin{flushright}
	{\sffamily The Director\\
		The \TeX nical Institute}
\end{flushright}
\section{Dividing the document}
\subsection{Example}
In this example, we show how subsections and subsubsections
are produced (there are no subsubsubsections). Note how the
subsections are numbered.
\subsubsection{Subexample}
Did you note that subsubsections are not numbered? This is so in the
\texttt{book} and \texttt{report} classes. In the \texttt{article}
class they too have numbers. (Can you figure out why?)
\paragraph{Note}
Paragraphs and subparagraphs do not have numbers. And they have
\textit{run-in} headings.
Though named ‘‘paragraph’’ we can have several paragraphs of text
within this.
\subparagraph{Subnote 1}
Subparagraphs have an additional indentation too.
\subparagraph{Subnote 2}
Subparagraphs have an additional indentation too.
\subparagraph{Subnote 3}
Subparagraphs have an additional indentation too.

\section{Example 2}
\subsection{Example}
In this example, we show how subsections and subsubsections
are produced (there are no subsubsubsections). Note how the
subsections are numbered.
\subsubsection{Subexample}
Did you note that subsubsections are not numbered? This is so in the
\texttt{book} and \texttt{report} classes. In the \texttt{article}
class they too have numbers. (Can you figure out why?)
\paragraph{Note}
Paragraphs and subparagraphs do not have numbers. And they have
\textit{run-in} headings.
Though named ‘‘paragraph’’ we can have several paragraphs of text
within this.
\subparagraph{Subnote 1}
Subparagraphs have an additional indentation too.
\subparagraph{Subnote 2}
Subparagraphs have an additional indentation too.
\subparagraph{Subnote 3}
Subparagraphs have an additional indentation too

\section{Tutorial III}
\subsection{Bibliography(Literaturverzeichnis)}
It is hard to write unstructured and disorganised documents using
\LaTeX˜\cite{les85}.It is interesting to typeset one
equation˜\cite[Sec 3.3]{les85} rather than setting ten pages of
running matter˜\cite{don89,rondon89}.\\If the author name is Alex and year 1991, the key can be coded as \textbf{\large{ale91}} or some
such mnemonic string.
\begin{thebibliography}{9}
	\bibitem{les85}Leslie Lamport, 1985. \emph{\LaTeX---A Document
		Preparation System---User’s Guide and Reference Manual},
	Addision-Wesley, Reading.
	\bibitem{don89}Donald E. Knuth, 1989. \emph{Typesetting Concrete
		Mathematics}, TUGBoat, 10(1):31-36.
	\bibitem{rondon89}Ronald L. Graham, Donald E. Knuth, and Ore
	Patashnik, 1989. \emph{Concrete Mathematics: A Foundation for
		Computer Science}, Addison-Wesley, Reading.
\end{thebibliography}

\section{Tutorial IV}
\subsection{Table of contents, index and glossary}
\contentsline{section}{4.Tutorial IV}{3}

\section{Tutorial V}
\subsection{Displayed Text}
\subsubsection{Borrowed words}
Some mathematicians elevate the spirit of Mathematics to a kind of
intellectual aesthetics. It is best voiced by Bertrand Russell in the
following lines.
\begin{quote}
	The true spirit of ................................from which
	all great work springs.
\end{quote}
To show it more clearly.
\subsubsection{Marking Lists}
One should keep the following in mind when using \TeX
{\renewcommand{\labelitemi}{$\blacktriangleright$}
	\begin{itemize}
	\item First main item with Black triangle right{
         \renewcommand{\labelitemii}{\ding{42}}
         \renewcommand{\labelitemiii}{\ding{43}}
         \renewcommand{\labelitemiv}{\ding{45}}
         	\begin{itemize}
         		\item The ding 42 in the first level         		
         		\item The ding 42 in the first level
         		\begin{itemize}
         			\item The ding43 in the second level
         			\item the ding43 in the second level
         			\begin{itemize}
         				\item The ding45 item in the third level
         				\item The ding45 item in the third level
         		    \end{itemize}
         	    \end{itemize}
             \end{itemize}}
	\item Second main item with Black triangle right
	\end{itemize}
Being a program, \TeX\ offers a high degree of flexibility.\\

\noindent Such a numbered list is produced by the enumerate environment in \LaTeX. The above list
was produced by the following source.
\begin{enumerate}
	\item prepare a source file with the extension "tex"
    \item Compile it with \LaTeX to produce a "dvi" file
    \item Print the document using a "dvi" driver
\end{enumerate}
The three basic steps in producing a printed document using \LaTeX\ are as follows:
\begin{enumerate}
	\item  First Item
	\item  Second Item
	\begin{enumerate}
		\item Use a previewer 
		\item Edit the source if needed
		\item Recompile
	\end{enumerate}
	\item Third Item
\end{enumerate}
\subsubsection{Descriptions and Definitions}
There is a example for the definitions layout.
\begin{description}
	\item[\TeX] A typesetting program
	\item[Emacs] A text editor and also
	\item[AbiWord] A word processor
	\item[$l$]
\end{description}
The real number $l$ is the least upper bound of the
set $A$ if it satisfies the following conditions
\begin{enumerate}
	\item[(1)] $l$ is an upper bound of $A$
	\item[(2)] if $u$ is an upper bound of $A$, then $l\le u$
\end{enumerate}
The second condition is equivalent to
\begin{enumerate}
	\item[(2)$'$] If $a<l$, then $a$ is not an upper bound of $A$.
\end{enumerate}
Let’s review the notation
{\renewcommand{\labelitemi}{$\bullet$}
\begin{itemize}
	\item $(0,1)$ is an \emph{open} interval
	\item $[0,1]$ is a \emph{closed} interval
\end{itemize}}
\subsubsection{Rows and Columns}
Let’s take stock of what we’ve learnt
\begin{tabbing}
	\hspace{1cm}\= \textbf{AbiWord}\quad\= A word processor\\[5pt]
	\> \textbf{Emacs} \> A text editor\\[5pt]
	\> \textbf{\TeX} \> A typesetting program\\[5pt]
	\> $\mu_{max}$\> Maximal Reibwert\\[5pt]
	\> \textbf{T}$_{b}$ \> Brake Torque
\end{tabbing}
\begin{tabbing}
	\textbf{AbiWord}\quad\= : \= A word processor\kill\\
	\textbf{\TeX}\quad \> : \> A typesetting program\\[5pt]
	\textbf{Emacs}\quad \> : \> A text editor\\[5pt]
	\> \> \quad\= a programming environment\\[5pt]
	\> \> \> a mail reader\\[5pt]
	\> \> \> and a lot more besides\\[5pt]
	\textbf{AbiWord}\quad\> : \> A word processor
\end{tabbing}
The example below illustrates all the tabbing command we've discussed.
\begin{tabbing}
	Row 1 Column 1\hspace{2cm}
	\= Row 1 Column 2\\[5pt]
	\> Row 2 Column 2\hspace{1.5cm}\=Row 2 Column 3\+\+\\[5pt]
	Row 3 Column 3\-\\[5pt]
	Row 4 Column 2 \>Row 4 Column 3\\[5pt]
	\< Row 5 Column 1 \> Row 5 Column 2 \>Row 5 Column 3\\[5pt]
	Row 6 Column 2 \>Row 6 Column 3\-\\[5pt]
	Row 7 Column 1 \> Row 7 Column 2 \>Row 7 Column 3\\[5pt]
	Row 8 Column 1 \ Right\\[5pt]
	Row 9 Column 1 \> and\ Row 9 Column 2\\[5pt]
	\pushtabs
	\quad\= Row 10 New Column 1\hspace{2.5cm}\= Row 10 New Column 2\\[5pt]
	\> Row 11 New Column 2 \> Row 11 New Column 2\\[5pt]
	\poptabs
	Row 12 Old Column 1\> Row 12 Old Column 2\>Row 12 Old Column 3
\end{tabbing}
\textbf{Tables}
\noindent The table below shows the sizes of the planets of our solar system.The {lr} specification immediately after the begin{tabular} indicates there are two columns in the table with the entries in the first column aligned on the left and the entries in the second column aligned on the right.\\
\textbf{lcc} means 3 Columns = left, center, center.
\begin{center}
	\begin{tabular}{lcc}
		Planet & Diameter(km) & Mass(T)\\[5pt]
		Mercury & 4878 & 1\\
		Venus & 12104 & 2\\
		Earth & 12756 & 3\\
		Mars & 6794 & 4 \\
		Jupiter & 142984 & 5\\
		Saturn & 120536 & 6\\
		Uranus & 51118 & 7\\
		Neptune & 49532 & 8\\
		Pluto & 2274 & 9
	\end{tabular}
\end{center}
As can be seen, Pluto is the smallest and Jupiter the largest\\
Here is another example.
\begin{center}
	\begin{tabular}{lp{.8\linewidth}}
		Planet & Features\\[5pt]
		Mercury & Lunar like crust, crustal faulting, small magnetic
		fields.\\
		Venus & Shrouded in clouds, undulating surface with highlands,
		plains, lowlands and craters.\\
		Earth & Ocens of water filling lowlands between continents,
		unique in supporting life, magnetic field.\\
		Mars & Cratered uplands, lowland plains, volcanic regions.\\
		Jupiter & Covered by clouds, dark ring of dust, magnetic field.\\
		Saturn & Several cloud layers, magnetic field, thousands
		of rings.\\
		Uranus & Layers of cloud and mist, magentic field, some rings.\\
		Neptune & Unable to detect from earth.\\
		Pluto & Unable to detect from earth
	\end{tabular}
\end{center}
Or another way:
\begin{center}
	\begin{tabular}{|l|c|c|}
		\hline
		Planet & Diameter(km) & Mass\\
		\hline
		Mercury & 4878 & 1\\
		Venus & 12104 & 2\\
		Earth & 12756 & 3\\
		Mars & 6794 & 4 \\
		Jupiter & 142984 & 5\\
		Saturn & 120536 & 6\\
		Uranus & 51118 & 7\\
		Neptune & 49532 & 8\\
		Pluto & 2274 & 9\\
		\hline
	\end{tabular}
\end{center}
Here the first few lines and the last lines of the input are as below (the other lines are the same as in the previous example).multicolumn{2} means merge the last two columns. the cline{2-3} means draws a horizontal line from the 2nd to 3rd column. cline{i-j} draws a horizontal line from the i$^{th}$ column to the j$^{th}$ column.
\begin{center}
	\begin{tabular}{|l|r|r|r|r|}
		\hline
		\multirow{2}{1.5cm}{\centering Planets} & \multicolumn{2}{c|}{Distance from sun (km)} & \multicolumn{2}{c|}{Mass from sun (km)}\\
		\cline{2-3}           \cline{4-5}
		& Maximum & Minimum & Maximum & Minimum\\
    	\hline
		Mercury & 4878 & 1 & 2 & 3\\
		Venus  & 12104 & 2 & 3 & 2\\
		Earth & 12756 & 3  & 1& 5\\
		Mars & 6794 & 4    & 5 & 88\\
		Jupiter & 142984 & 5 & 1 & 99\\
		Saturn & 120536 & 6 & 3 & 21\\
		Uranus & 51118 & 7 & 2 & 2\\
		Neptune & 49532 & 8 & 8 & 0\\
		Pluto & 2274 & 9 & 1 & 2\\
		\hline
	\end{tabular}
\end{center}
Next comes th \textbf{Array package}: The m{wd} specifier produces a column of width wd just like the p specifier, but with the text aligned vertically in the middle unlike the p specifier which aligns the text with the topline. (The table on the left, incidently, was produced by the same input as above but with p instead of m).\\
\begin{center}
	\begin{tabular}{|>{\bfseries}l|r|}
		\hline
		\multicolumn{1}{|m{1.5cm}|}{\centering Planet} &\multicolumn{1}{m{2.3cm}|}{\centering Mean distance from sun \\ (km)}\\
		\hline
		Mercury & 4878\\
		Venus  & 12104\\
		Earth & 12756\\
		Mars & 6794 \\
		Jupiter & 142984\\
		Saturn & 120536\\
		Uranus & 51118\\
		Neptune & 49532\\
		Pluto & 2274\\
		\hline
	\end{tabular}
\end{center}
\begin{center}
	\begin{tabular}{|>{\bfseries}l|r|r|}
		\hline
		\multirow{3}{1.5cm}{\centering Planet}
		& \multicolumn{2}{p{3.5cm}|}
		{\centering Distance from sun \\ (million km)}\\
		\cline{2-3}
		& \multicolumn{1}{c|}{Maximum}
		& \multicolumn{1}{c|}{Minimum}\\
		\hline
		Mercury & 69.4 & 46.8\\
		Venus & 109.0 & 107.6\\
		Earth & 152.6 & 147.4\\
		Mars & 249.2 & 207.3\\
		Jupiter & 817.4 & 741.6\\
		Saturn & 1512.0 & 1346.0\\
		Uranus & 3011.0 & 2740.0\\
		Neptune & 4543.0 & 4466.0\\
		Pluto & 7346.0 & 4461.0\\
		\hline
	\end{tabular}
\end{center}
\newpage

\section{Tutorial VI}
\subsection{Typesetting Mathematics}
\subsubsection{The Basics}
A mathematical expression occurring in running text (called in-text math) is produced by enclosing it between dollar signs. Thus to produce. The equation representing a straight line in the Cartesian plane is of the form $ax+by+c=0$, where $a$, $b$, $c$ are constants.\\
This can be done by changing the input as follows:
The equation representing a straight line in the Cartesian plane is
of the form
$$
ax+by+c=0
$$
where $a$, $b$, $c$ are constants.
Again \$\$ ax+by+c=0 \$\$ is the \TeX \ way of producing displayed math. LATEX has the constructs
\[ a+b+c=0\] or \begin{displaymath} a+b+c = 0 \end{displaymath} also to do this.\\
It is easily seen that $(x^m)^n=x^{mn}$.\\
The sequence $(x_n)$ defined by
$$
x_1=1,\quad x_2=1,\quad x_n=x_{n-1}+x_{n-2}\;\;(n>2)
$$
is called the Fibonacci sequence.
$$
x_m^n\qquad x^n_m\qquad {x_m}^n\qquad {x^n}_m
$$
The sequence
$$
2\sqrt{2}\,,\quad 2^2\sqrt{2-\sqrt{2}}\,,\quad 2^3
\sqrt{2-\sqrt{2+\sqrt{2}}}\,,\quad 2^4\sqrt{2-
	\sqrt{2+\sqrt{2+\sqrt{2+\sqrt{2}}}}}\,,\;\ldots
$$
converge to $\pi$.\\
\\Next is Custom Commands:\\
\newcommand{\vect}[2]{(#1_1,#1_2,\dots,#1_#2)}
\begin{displaymath} \vect{x}{n},\\ \vect{y}{m},\\ \vect{a}{p} \end{displaymath}\\
The equation representing a straight line in the Cartesian plane is
of the form
\begin{equation}
	ax+by+c=0\tag{1-1}
\end{equation}
where $a$, $b$, $c$ are constants.
\begin{equation}
    ax+by+c=1
\end{equation}
\begin{subequations}
	Maxwell's equations:
	\begin{align}
	B'&=-\nabla \times E,\\
	E'&=\nabla \times B - 4\pi j,
	\end{align}
\end{subequations}
\[
z = \overbrace{
	\underbrace{x}_\text{real} + i
	\underbrace{y}_\text{imaginary}
}^\text{complex number}
\]
\textbf{https://en.wikibooks.org/wiki/LaTeX/Advanced\_Mathematics}
\begin{align*}
\cosˆ2x+\sinˆ2x & = 1 & \coshˆ2x-\sinhˆ2x & = 1\\
\cosˆ2x-\sinˆ2x & = \cos 2x & \coshˆ2x+\sinhˆ2x & = \cosh 2x
\end{align*}

\begin{equation}
\begin{aligned}
\cosˆ2x+sinˆ2x & = 1\\
\cosˆ2x-\sinˆ2x & = \cos 2x
\end{aligned}
\qquad\text{and}\qquad
\begin{aligned}
\coshˆ2x-\sinhˆ2x & = 1\\
\coshˆ2x+\sinhˆ2x & = \cosh 2x
\end{aligned}
\end{equation}

\begin{equation}
|x| =
\begin{cases}
x & \text{if $x\ge 0$}\\
-x & \text{if $x\le 0$}
\end{cases}
\end{equation}\\
\subsubsection{The Matrix}
The system of equations:
\begin{align}
x+y-z & = 1\\
x-y+z & = 1\\
x+y+z & = 1
\end{align}
can be written in matrix terms as
\begin{equation}
\begin{pmatrix}
1 & 1 & -1\\
1 & -1 & 1\\
1 & 1 & 1
\end{pmatrix}
\begin{pmatrix}
x\\
y\\
z
\end{pmatrix}
=
\begin{pmatrix}
1\\
1\\
1
\end{pmatrix}.
\end{equation}
Here, the matrix
$\begin{pmatrix}
1 & 1 & -1\\
1 & -1 & 1\\
1 & 1 & 1
\end{pmatrix}$
is invertible.\\
A general $m\times n$ matrix is of the form:
\begin{equation}
\begin{pmatrix}
a_{11} & a_{12} & \dots & a_{1n}\\
a_{21} & a_{22} & \dots & a_{2n}\\
\vdots & \vdots & \dots & \vdots \\
a_{m1} & a_{m2} & \dots & a_{mn}
\end{pmatrix}
\end{equation}
\begin{equation*}
\left.
\begin{aligned}
u_x & = v_y\\
u_y & = -v_x
\end{aligned}
\right\}
\quad\text{Cauchy-Riemann Equations}
\end{equation*}\\
\begin{equation}
(x+y)ˆ2-(x-y)ˆ2=\left((x+y)+(x-y)\right)\left((x+y)-(x-y)\right)=4xy
\end{equation}\\

\begin{equation*}
\left(\sum_{k=1}^n|x_ky_k|\right)^2\le
\left(\sum_{k=1}^{n}|x_k|\right)\left(\sum_{k=1}^{n}|y_k|\right)
\end{equation*}\\
For $n$-tuples of complex numbers $(x_1,x_2,\dotsc,x_n)$ and
$(y_1,y_2,\dotsc,y_n)$ of complex numbers
\begin{equation*}
\biggl(\sum_{k=1}^n|x_ky_k|\biggr)^2\le
\biggl(\sum_{k=1}^{n}|x_k|\biggr)\biggl(\sum_{k=1}^{n}|y_k|\biggr)
\end{equation*}\\
\begin{equation}
1-\binom{n}{1}\frac{1}{2}+\binom{n}{2}\frac{1}{2^2}-\dotsb
-\binom{n}{n-1}\frac{1}{2^{n-1}}=0
\end{equation}\\
Since $(x_n)$ converges to $0$, there exists a positive integer $p$
such that
\begin{equation*}
|x_n|<\tfrac{1}{2}\quad\text{for all $n\ge \pi$}
\end{equation*}\\
\newcommand{\abc}[2]{\genfrac{[}{]}{0pt}{}{#1}{#2}}
\newcommand{\chssk}[2]{\genfrac{\{}{\}}{0pt}{}{#1}{#2}}
The Christoffel symbol $\genfrac{\{}{\}}{0pt}{}{ij}{k}$ of the second
kind is related to the Christoffel symbol $\genfrac{[}{]}{0pt}{}{ij}{k}$
of the first kind by the equation
\begin{equation*}
\chssk{ij}{k}=g^{k1}\abc{ij}{1}+g^{k2}\abc{ij}{2}
\end{equation*}\\
Thus we get:
\begin{equation*}
0\xrightarrow{} A\xrightarrow[\text{monic}]{f}
B\xrightarrow[\hspace{7pt}\text{epi}\hspace{7pt}]{g}
C\xrightarrow{}0
\end{equation*}\\
Euler not only proved that the series
$\displaystyle\sum_{n=1}^\infty\frac{1}{n^2}$ converges, but also that $\sum_{n=1}^\infty\frac{1}{n^2}$
\begin{equation*}
\sum_{n=1}^\infty\frac{1}{n^2}=\frac{\pi^2}{6}
\end{equation*}\\
Thus
$\lim\limits_{x\to\infty}\int_0^x\frac{\sin x}{x}\,\mathrm{d}x
=\frac{\pi}{2}$
and so by definition,
\begin{align*}
\int_0^\infty\frac{\sin x}{x}\,\mathrm{d}x=\frac{\pi}{2}
\qquad\text{and}\qquad
\int\limits_0^\infty\frac{\sin x}{x}\,\mathrm{d}x=\frac{\pi}{2}
\end{align*}\\
There is another reason for tweaking the math fonts. Recently, the International
Standards Organization (ISO) has established the recognized typesetting standards in
mathematics. Some of the points in it are,\\
1. Simple variables are represented by italic letters as a, x.\\
2. Vectors are written in boldface italic as $a, x$.\\
3. Matrices may appear in sans serif as in $\mathbf{A,X}$.\\
4. The special numbers e, i and the differential operator d are written in\ \textit{upright } roman. For example, \
\newcommand{\me}{\mathrm{e}}$\me^2,$
\newcommand{\mi}{\mathrm{i}}$\mi,$
\newcommand{\diff}{\mathrm{d}}$\diff x$.

\newpage
\section{Tutorial VII}
\subsection{Typesetting Theorems}
In Mathematical documents we often have special statements such as \textit{axioms}(which are nothing but the assumptions made) and \textit{theorems}(which are the conclusions obtained, sometimes known by ohter names like propositions or lemmas). These are often typeset in different font to distinguish them from surrounding text and given a name and a number for subsequent reference.
\begin{thm}[Euclid]
	The sum of the angles of a triangle is $180^\circ$.
\end{thm}
\begin{thm}[Ly]
	The sum of the angles of a normal triangle is $180^\circ$.
\end{thm}
\begin{dfn}
	A triangle is the figure formed by joining each pair
	of three non collinear points by line segments.
\end{dfn}
\begin{thm}[Euclid]
	The number of primes is infinite.
\end{thm}
\begin{proof}
	This follows easily from the equation
	\begin{equation}
	(x+y)ˆ2=xˆ2+yˆ2+2xy\tag*{\qed}
	\end{equation}
	\renewcommand{\qed}{}
\end{proof}

\newpage
\section{Tutorial VIII}
\subsection{Several Kinds of Boxes}
\framebox{A few words of advice}\\[6pt]
\fbox{Text in a box}
\setlength\fboxrule{2pt}\setlength\fboxsep{2mm}
\fbox{Text in a box}

\newpage
\section{Tutorial IX}
\subsection{Floats}
Figures are really problematical to present in a document because they never split between pages. This leads to bad page breaks which in turn leave blank space at the bottom of pages. For fine-tuning that document, the typesetter has to adjust the page breaks manually.
But \LaTeX\ provides floating figures which automatically move to suitable locations.
So the positioning of figures is the duty of \LaTeX.\\ 
Floating figures are created by putting commands in a figure environment. The contents of the figure environment always remains in one chunk, floating to produce good page breaks. The following commands put the graphic from figure.eps inside a floating
figure:\\
\begin{figure}[htbp]
	\centering
	\includegraphics[width=6.5cm]{picture/tex.png}
	\caption{Tex Users Group}
	\label{fig:tex1}
\end{figure}\\
\newpage
\noindent Anotherway \LaTeX\ provides floating figures which automatically move to suitable locations.
So the positioning of figures is the duty of \LaTeX.\\ 
Floating figures are created by putting commands in a figure environment. The contents of the figure environment always remains in one chunk, floating to produce good page breaks. The following commands put the graphic from figure.eps inside a floating
figure:\\
\begin{figure}[htbp]
	\centering
	\subfigure[the 1st subfigure]{
		\begin{minipage}[b]{0.2\textwidth}
			\includegraphics[width=1\textwidth]{picture/tex1.png} \\
			\includegraphics[width=1\textwidth]{picture/tex2.png}
		\end{minipage}
	}
	\subfigure[the 2rd subfigure]{
		\begin{minipage}[b]{0.2\textwidth}
			\includegraphics[width=1\textwidth]{picture/tex1.png} \\
			\includegraphics[width=1\textwidth]{picture/tex2.png}
		\end{minipage}
	}
\end{figure}


































































\end{document}